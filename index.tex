% Options for packages loaded elsewhere
\PassOptionsToPackage{unicode}{hyperref}
\PassOptionsToPackage{hyphens}{url}
\PassOptionsToPackage{dvipsnames,svgnames,x11names}{xcolor}
%
\documentclass[
  letterpaper,
  DIV=11,
  numbers=noendperiod]{scrreprt}

\usepackage{amsmath,amssymb}
\usepackage{iftex}
\ifPDFTeX
  \usepackage[T1]{fontenc}
  \usepackage[utf8]{inputenc}
  \usepackage{textcomp} % provide euro and other symbols
\else % if luatex or xetex
  \usepackage{unicode-math}
  \defaultfontfeatures{Scale=MatchLowercase}
  \defaultfontfeatures[\rmfamily]{Ligatures=TeX,Scale=1}
\fi
\usepackage{lmodern}
\ifPDFTeX\else  
    % xetex/luatex font selection
\fi
% Use upquote if available, for straight quotes in verbatim environments
\IfFileExists{upquote.sty}{\usepackage{upquote}}{}
\IfFileExists{microtype.sty}{% use microtype if available
  \usepackage[]{microtype}
  \UseMicrotypeSet[protrusion]{basicmath} % disable protrusion for tt fonts
}{}
\makeatletter
\@ifundefined{KOMAClassName}{% if non-KOMA class
  \IfFileExists{parskip.sty}{%
    \usepackage{parskip}
  }{% else
    \setlength{\parindent}{0pt}
    \setlength{\parskip}{6pt plus 2pt minus 1pt}}
}{% if KOMA class
  \KOMAoptions{parskip=half}}
\makeatother
\usepackage{xcolor}
\setlength{\emergencystretch}{3em} % prevent overfull lines
\setcounter{secnumdepth}{5}
% Make \paragraph and \subparagraph free-standing
\ifx\paragraph\undefined\else
  \let\oldparagraph\paragraph
  \renewcommand{\paragraph}[1]{\oldparagraph{#1}\mbox{}}
\fi
\ifx\subparagraph\undefined\else
  \let\oldsubparagraph\subparagraph
  \renewcommand{\subparagraph}[1]{\oldsubparagraph{#1}\mbox{}}
\fi


\providecommand{\tightlist}{%
  \setlength{\itemsep}{0pt}\setlength{\parskip}{0pt}}\usepackage{longtable,booktabs,array}
\usepackage{calc} % for calculating minipage widths
% Correct order of tables after \paragraph or \subparagraph
\usepackage{etoolbox}
\makeatletter
\patchcmd\longtable{\par}{\if@noskipsec\mbox{}\fi\par}{}{}
\makeatother
% Allow footnotes in longtable head/foot
\IfFileExists{footnotehyper.sty}{\usepackage{footnotehyper}}{\usepackage{footnote}}
\makesavenoteenv{longtable}
\usepackage{graphicx}
\makeatletter
\def\maxwidth{\ifdim\Gin@nat@width>\linewidth\linewidth\else\Gin@nat@width\fi}
\def\maxheight{\ifdim\Gin@nat@height>\textheight\textheight\else\Gin@nat@height\fi}
\makeatother
% Scale images if necessary, so that they will not overflow the page
% margins by default, and it is still possible to overwrite the defaults
% using explicit options in \includegraphics[width, height, ...]{}
\setkeys{Gin}{width=\maxwidth,height=\maxheight,keepaspectratio}
% Set default figure placement to htbp
\makeatletter
\def\fps@figure{htbp}
\makeatother
\newlength{\cslhangindent}
\setlength{\cslhangindent}{1.5em}
\newlength{\csllabelwidth}
\setlength{\csllabelwidth}{3em}
\newlength{\cslentryspacingunit} % times entry-spacing
\setlength{\cslentryspacingunit}{\parskip}
\newenvironment{CSLReferences}[2] % #1 hanging-ident, #2 entry spacing
 {% don't indent paragraphs
  \setlength{\parindent}{0pt}
  % turn on hanging indent if param 1 is 1
  \ifodd #1
  \let\oldpar\par
  \def\par{\hangindent=\cslhangindent\oldpar}
  \fi
  % set entry spacing
  \setlength{\parskip}{#2\cslentryspacingunit}
 }%
 {}
\usepackage{calc}
\newcommand{\CSLBlock}[1]{#1\hfill\break}
\newcommand{\CSLLeftMargin}[1]{\parbox[t]{\csllabelwidth}{#1}}
\newcommand{\CSLRightInline}[1]{\parbox[t]{\linewidth - \csllabelwidth}{#1}\break}
\newcommand{\CSLIndent}[1]{\hspace{\cslhangindent}#1}

\KOMAoption{captions}{tableheading}
\makeatletter
\makeatother
\makeatletter
\@ifpackageloaded{bookmark}{}{\usepackage{bookmark}}
\makeatother
\makeatletter
\@ifpackageloaded{caption}{}{\usepackage{caption}}
\AtBeginDocument{%
\ifdefined\contentsname
  \renewcommand*\contentsname{Table of contents}
\else
  \newcommand\contentsname{Table of contents}
\fi
\ifdefined\listfigurename
  \renewcommand*\listfigurename{List of Figures}
\else
  \newcommand\listfigurename{List of Figures}
\fi
\ifdefined\listtablename
  \renewcommand*\listtablename{List of Tables}
\else
  \newcommand\listtablename{List of Tables}
\fi
\ifdefined\figurename
  \renewcommand*\figurename{Figure}
\else
  \newcommand\figurename{Figure}
\fi
\ifdefined\tablename
  \renewcommand*\tablename{Table}
\else
  \newcommand\tablename{Table}
\fi
}
\@ifpackageloaded{float}{}{\usepackage{float}}
\floatstyle{ruled}
\@ifundefined{c@chapter}{\newfloat{codelisting}{h}{lop}}{\newfloat{codelisting}{h}{lop}[chapter]}
\floatname{codelisting}{Listing}
\newcommand*\listoflistings{\listof{codelisting}{List of Listings}}
\makeatother
\makeatletter
\@ifpackageloaded{caption}{}{\usepackage{caption}}
\@ifpackageloaded{subcaption}{}{\usepackage{subcaption}}
\makeatother
\makeatletter
\@ifpackageloaded{tcolorbox}{}{\usepackage[skins,breakable]{tcolorbox}}
\makeatother
\makeatletter
\@ifundefined{shadecolor}{\definecolor{shadecolor}{rgb}{.97, .97, .97}}
\makeatother
\makeatletter
\makeatother
\makeatletter
\makeatother
\ifLuaTeX
  \usepackage{selnolig}  % disable illegal ligatures
\fi
\IfFileExists{bookmark.sty}{\usepackage{bookmark}}{\usepackage{hyperref}}
\IfFileExists{xurl.sty}{\usepackage{xurl}}{} % add URL line breaks if available
\urlstyle{same} % disable monospaced font for URLs
\hypersetup{
  pdftitle={CASA0023 - Personal Diary},
  pdfauthor={Ana Steiner},
  colorlinks=true,
  linkcolor={blue},
  filecolor={Maroon},
  citecolor={Blue},
  urlcolor={Blue},
  pdfcreator={LaTeX via pandoc}}

\title{CASA0023 - Personal Diary}
\author{Ana Steiner}
\date{2024-01-23}

\begin{document}
\maketitle
\ifdefined\Shaded\renewenvironment{Shaded}{\begin{tcolorbox}[interior hidden, enhanced, boxrule=0pt, borderline west={3pt}{0pt}{shadecolor}, frame hidden, sharp corners, breakable]}{\end{tcolorbox}}\fi

\renewcommand*\contentsname{Table of contents}
{
\hypersetup{linkcolor=}
\setcounter{tocdepth}{2}
\tableofcontents
}
\bookmarksetup{startatroot}

\hypertarget{about}{%
\chapter{About}\label{about}}

Hello to all. I am Ana :)

First of all, I am a proud Colombian woman, very interested in
understanding socio-cultural, economic and environmental behaviors
around me, and commited to improving livelyhoods and conditions for
people and communities around the world.

I studied anthropology for my undergrad and later finished an MSc in
Sustainable Development, currently I am studying an MSc in Social and
Geographic Data Science. I have more than 8 years of experience working
in rural development in Colombia, analyzing and developing different
economic activities that contribute to both cultural and environmental
conservation, as well as coordinating efforts with regional governments,
NGO's and the environmental sector.

My expectation with this masters, and this specific course, is to
adquire as many tools and skills in social and environmental data
collection and analysis, that will permit me to continue designing,
implementing and evaluating development projects and programs.

\bookmarksetup{startatroot}

\hypertarget{section}{%
\chapter{}\label{section}}

Week 1

Introduction to Remote Sensing

NASA defines remote sensing as ``acquiring information from a
distance''. This information is acquired through sensors mounted on
different platforms such as satellites, planes, drones, etc\ldots{}
Currently there are more than 150 satellites and more than 27,000 other
devices orbiting the earth.

There are various types of sensors and data sources, as well as
applications. Data can come from Sentinel, LANDSAT or SAR satellites,
and can be used to monitor various phenomenons such as land covered
areas, urban or green space coverage, natural and climate phenomenons,
and other atmospheric characteristics.

Types of sensors

There are two types of sensors:

Passive sensors use and detect energy that is available from the sun,
but dont emit their own. Examples of this type are the human eye or a
camera.

Active sensors have proper sources of illumination and actively emit
eletromagnetic waves. Some examples are X-rays and Lidar sensors. These
are appropriate for difficult climatology and atmospheric conditions
such as clouds and storms or air pollution, or to see during the night
sky.

Reflection and Resolutions

Natural or unnatural phenomenon, such as clouds or volcanic ash, can
interrupt or affect the reflection and dispersion of electromagnetic
waves. The earth absorbs some energy, the other is randomly dispersed in
the atmosphere. Different sensors (as mentioned above) are used to
identify and capture different electromagnetic wave lengths and types.

Independent of the sensor used, 4 resolutions are needed:

**Spectral:** The number of bands recorded (red, green, blue for visible
light).

**Spatial:** The size of each pixel (from cm to km depending on the
scope of analysis).

**Temporal:** The frequency with given geographical area is monitored.

**Radiometric:** The range of reflectance values that are registered.

Practical application and use

Remote sensing, as it name implies, is used to monitor and register
spatial data from a remote sensor such as satellites; it is a great
compliment to more traditional and smaller-scale on-site monitors which
only recover very localized information. I make an enphasis on
``compliment'' because it also has limitations that are worth
mentioning, such as its cost, resolution limitations in some cases, and
the need for a more robust data processing and analysis.

Remote sensing can be used in various sectors such as agriculture,
climate change mitigation and environmental conservation, monitoring of
natural hazzards, meteorology, as well as urban planning and
development. Personally, I find it fascinated how it can be used to
visualize and monitor natural resource management (conservation of
hydric resources, of forest coverage, etc.).

In my view, the most exciting applications for remote sensing data are
in situations where traditional forms of data collection -\/- such as a
national census -\/- are unreliable or out of date. This is true of
South Africa, where obtaining up-to-date demographic or economic
information is very difficult. I'll discuss that a bit more in week 4,
when I get into policy.

All though there are many applications, the ones that were covered in
class are:

Sentinel Data - It comes from the European Earth observation satellite
program, specifically from satellites that capture high-resolution
optical and radar imagery. They use various sensors such as MSI for
optical imagery and SAR for radar data; the data recollected is mostly
used for environmental monitoring, disaster management and agricultural
monitoring.

Landsat Data - It comes from the NASA and the United States Geological
Survey, specifically from satellites that have been orbiting for more
than 50 years. These sensors capture different types of spectral data
(visible, infrared and thermal, for example). This data is used for land
cover monitoring, land use management and environmental monitoring. Due
to its long-term archive, it's data has been very useful for studying
environmental and urban developments and evolution.

Personal reflection

This is the first time that I work with this type of data. To be honest,
it is quite an overwhelming process. It is very technical, I have had to
review basic physics that I had not used in many years to remember the
electromagnetic spectrum, the different types of wave lengths, colors,
etc. It has also been a bit difficult to download data that works and
then visualizing it on QGis or Rstudio, for me it is a work in progress
as I learn how to be more efficient and accurate.

It is quite amazing to see how these large satellites, that I have
always seen as very scientific and foreign to me, can produce images and
data that is super important and useful for my ``real world''
applications such as conservation monitoring. Coming from Colombia, a
country with unreliable and incomplete environmental and demographic
data, I feel that these tools can be amazingly useful to have real-time
ecological and social monitoring.

I am excited to see how the course progresses and all that I can learn
and later apply.

\bookmarksetup{startatroot}

\hypertarget{section-1}{%
\chapter{}\label{section-1}}

Week 2

I share with you this presentation I made on Landsat satellites: their
great importance to modern science, the methodologies and technology
they use, their different applications and improvements over time:

:::: xaringan source: C:/Users/USUARIO/Documents/UCL/Term
2/CASA0023/week 2/xaringan\_presentation.Rmd

Enjoy!

\bookmarksetup{startatroot}

\hypertarget{section-2}{%
\chapter{}\label{section-2}}

Week 3

\textbf{Geometric and Atmospheric Correction:}

\begin{itemize}
\item
  Satellite images are assigned a coordinate reference system (CRS) in
  GIS.
\item
  Image distortions can result from factors like off-nadir view angles,
  topography, wind (in aerial data collection), and Earth's rotation.
\item
  Geometric correction aims to rectify these distortions for accurate
  spatial analysis.
\item
  Ground Control Points (GPS) are used to match known points in the
  image with a reference dataset or map.
\item
  Coordinates from GPS points are used to model geometric transformation
  coefficients.
\item
  Various transformation algorithms are available for modeling the
  actual coordinates.
\item
  The correction process involves both forward and backward mapping to
  ensure alignment.
\item
  Root Mean Square Error (RMSE) is used to assess the accuracy of the
  correction.
\item
  Lower RMSE values indicate a better fit of corrected data.
\item
  Resampling methods like Nearest Neighbor, Linear, Cubic, and Cubic
  spline may be applied.
\item
  Environmental factors, including atmospheric scattering and
  topographic attenuation, can affect remote sensing data.
\item
  Atmospheric correction is essential when precise reflectance values
  are required for analysis.
\item
  It involves normalizing intensities of different bands within an image
  or across multiple dates.
\item
  Methods for atmospheric correction include relative, pseudo-invariant
  features (PIFs), and absolute correction.
\item
  Absolute correction requires atmospheric measurements and radiative
  transfer models.
\item
  Tools like ACORN, FLAASH, QUAC, and ATCOR are available for
  atmospheric correction.
\item
  Radiometric calibration converts digital brightness values (DN) to
  scaled surface reflectance.
\item
  Landsat data is often distributed as a surface reflectance product.
\item
  Atmospheric correction aims to remove the effects of atmospheric
  scattering and absorption.
\item
  The corrected data provides more accurate information for analysis and
  interpretation.
\end{itemize}

\textbf{Practical Application and Use:}

\begin{itemize}
\item
  These correction techniques are crucial for obtaining accurate and
  reliable data in remote sensing.
\item
  They ensure that imagery can be effectively utilized for various
  applications.
\item
  Corrected data is essential for tasks like land cover classification
  and change detection.
\item
  It enables the extraction of valuable information from remote sensing
  imagery.
\item
  Practical applications include agriculture, urban planning,
  environmental monitoring, and disaster management.
\item
  Atmospheric correction is particularly important for tasks involving
  precise reflectance values.
\item
  It supports the assessment of biophysical parameters like temperature,
  leaf area index, and NDVI.
\item
  Corrected data enhances the quality and usefulness of remote sensing
  products.
\item
  They are integral in providing decision-makers with valuable insights
  for informed choices.
\end{itemize}

\textbf{Personal Reflection:}

\begin{itemize}
\item
  The understanding and application of geometric and atmospheric
  correction are fundamental in the field of remote sensing.
\item
  These correction processes ensure that the data accurately represents
  the Earth's surface.
\item
  As technology advances, new tools and algorithms continue to improve
  the correction accuracy.
\item
  Remote sensing offers a powerful means of gathering information about
  our planet.
\item
  The application of these techniques to urban environments can have a
  profound impact on decision-making and urban planning.
\item
  The combination of practical skills and theoretical knowledge in
  remote sensing is crucial for professionals in the field.
\item
  I recognize the importance of continually learning and adapting to new
  advancements in remote sensing technology.
\item
  Geometric and atmospheric correction lay the foundation for reliable
  and insightful remote sensing analysis.
\end{itemize}

\bookmarksetup{startatroot}

\hypertarget{section-3}{%
\chapter{}\label{section-3}}

Week 4 - Public Policy Recommendations

Case study: Bogota's Green Corridor Plan

\begin{itemize}
\tightlist
\item
  Context of Bogota
\end{itemize}

Bogota is a city heavily criticized for its poor urban planning; it is
one of the largest and densest capital cities in Latin America yet it
suffers from having some of the worst public transportation systems and
a lack of sustainable infrastructure \textbf{(cita).}

For many decades, Bogota was one of the most advanced and innovative
cities, home to Transmilenio (xx) and at the time, the largest bicycle
route in South America (xx), as well as having constructed various
sustainable buildings (xx). Nonetheless, the city has grown at such a
fast rate, and in such an informal way, that the urban planning has not
been able to support the growing demographic necessities.
\textbf{(cita).}

Because of this, it's inhabitants are exposed to heavily contaminated
air, to horrible traffic, to an expensive transportation system that
does not support the population density of the city, and to great
disparities regarding access to sustainable infrastructure such as as
green housing, bicycle routs, etc\ldots{} \textbf{(cita).} It is one of
few cities with more than 7 million citizens that does NOT have a metro
system. \textbf{(cita)}, making it very dependent on cars. There have
been many proposals for urban ``retenes'' and carpooling obligations,
which have all been debated and blocked by congress.

\begin{itemize}
\tightlist
\item
  Septima Verde
\end{itemize}

One of the newest project ideas is the Septima Verde Project, which
proposes to build a green corridor along one of the largest and busiest
avenues in the city; this project includes building pedestrian and cycle
friendly routes and the delimitation of green spaces and parks,
ultimately reducing car lanes and hopefully encouraging for more
non-motorized transportation. The project's goal is also to promote more
sustainable urban development plans and improve air quality.

This project will use the current sidewalk and 4 vehicule lanes (half of
the highway) as well as various unused lots and dwellings.

\includegraphics{.quarto/CorredorVerde3-1.webp}

Objectives:

\begin{enumerate}
\def\labelenumi{\arabic{enumi}.}
\item
  To transform certain streets and corridors into pedestrian-friendly
  spaces with an emphasis on greenery, parks, and public spaces.
\item
  To reduce car lanes and allocate more space for pedestrians, cyclists,
  and public transportation.
\item
  To encourage the use of non-motorized transportation, such as walking
  and cycling.
\item
  To combat air pollution and enhance urban sustainability by increasing
  green spaces and reducing car emissions.
\end{enumerate}

Key Components:

\begin{enumerate}
\def\labelenumi{\arabic{enumi}.}
\item
  Urban Planning: The initiative involves urban planning and redesign
  efforts to allocate more space for pedestrians, greenery, and public
  amenities.
\item
  Cycling Infrastructure: The creation of dedicated bike lanes and
  cycling infrastructure to promote sustainable transportation options.
\item
  Public Transportation Integration: Integration with public
  transportation systems, such as TransMilenio, to encourage the use of
  buses and reduce private car use.
\item
  Green Spaces: The inclusion of green spaces, parks, and urban gardens
  along the corridors to enhance the urban environment and improve air
  quality.
\item
  Accessibility: Improving accessibility for people with disabilities
  through the creation of barrier-free spaces.
\item
  Street Furniture: Installation of street furniture, benches, and
  public art to enhance the overall experience of the corridors.
\end{enumerate}

Within the project, it is also contemplated to construct new routs for
Transmilenio (Bogota's bus system) around this corridor, giving people
more incentives to not use their car. This last component of the project
has generated much public and political debate. The Mayoress of Bogota
had made a big point, during her campaign, to not building public
transportation infrastructure during this avenue, and this project would
be a direct contradiction to this, the main opposers of the project are
ironically her election base.

This makes for a very interesting political setting. The creation of the
green corridor, as well as the public transportation routes, would
really help some of the most deprived populations, as it would give them
access to better transportation, to more green areas, and in general
would hopefully decongest the traffic; but it has been made clear that
it would affect negatively the richer citizens who have benefited from
this highway with no public transportation and just private cars. This
project has become a clear example of how difficult it is to propose and
implement a public policy or a public project that keeps everyone
content. Additionally, many opposers also suggest better ways to use the
unused lots such as building schools or community centers.

Additionally, this is a hefty project budgeted at 2.5 billion COP (aprox
640 million USD).

\begin{itemize}
\tightlist
\item
  Applications of remote sensing:
\end{itemize}

I. During the planning and design of this project:

- Land Use Data (Sentinel-2, Landsat 8): To identify existing use
patterns, distinguishing between green spaces, industrial
infrastructure, housing, roads, etc. This was critical in defining the
green area baseline to which to aggregate too, as well as define the
businesses and residential neighborhoods that would be affected.

- Air Quality Data (Sentinel-5p): To monitor the baseline of air
pollution in the designated areas to define the benchmarks.

- Vegetation and Land Cover Monitoring (Sentinel-2, Landsat 8): To
characterize the current vegetation health status and biodiversity, and
to better define what type of green areas to add.

II. During the monitoring and compliance of the project:

- Urban Heat Monitoring (Thermal imagery): To measure the cooling impact
of the green spaces and possible reduced traffic.

- Air Quality Assessment (Sentinel-5p): To measure improvements in air
quality.

- Traffic Analysis (Satellite traffic data): To identify changes in
flows and mobility patterns, to measure traffic of pedestrian and of
vehicle paths, to identify consequences over traffic in surrounding
highways.

All of this will be complimented with robust demographic and socio
economic data, to measure improvements over economy, income and mobility
in the affected populations and sectors; as well as with data from
traffic police or public transportation companies regarding the hopeful
increase in people using public transportation and a reduction in car
use.

\begin{itemize}
\tightlist
\item
  Personal reflections:
\end{itemize}

As a Colombian and Bogotá-raised anthropologist and geographer deeply
passionate about my city's vibrant culture and potential for sustainable
development, my journey has been both inspiring and challenging. Bogotá,
with its rich history and diverse communities, has always held a special
place in my heart. Witnessing the city's progress towards sustainability
has been a source of hope and pride, as initiatives like ``Septima
Verde'' have aimed to transform our urban landscape into a greener, more
pedestrian-friendly haven.

However, my passion is tempered by concerns and a critical perspective.
The specter of corruption and inefficiency that has haunted local
governments often overshadows these noble efforts. It pains me to see
the misallocation of resources and missed opportunities for meaningful
change. My role as an anthropologist and geographer has fueled my
determination to not only celebrate our successes but also hold our
leaders accountable for their actions---or lack thereof.

Navigating the delicate balance between love for my city and the need
for reform has been a constant theme in my personal and professional
life. Despite the challenges, I remain dedicated to contributing to the
sustainable development of Bogotá and fostering a brighter future for
all its inhabitants, driven by my unwavering belief in the potential of
our beautiful city.

\begin{itemize}
\item
  Bibliography:

  https://elpais.com/america-colombia/2023-10-26/el-corredor-verde-de-la-septima-proyecto-insignia-de-claudia-lopez-naufraga-en-un-juzgado.html

  https://www.portafolio.co/economia/corredor-verde-por-la-septima-se-mantiene-en-que-consiste-el-proyecto-570477

  https://observatorio.dadep.gov.co/noticias/plan-para-transformar-la-carrera-7ma-en-bogota

  https://www.lasillavacia.com/red-de-expertos/red-cachaca/repensando-el-corredor-verde-de-la-septima-estrategias-temporales-para-predios-en-espera/
\end{itemize}

\bookmarksetup{startatroot}

\hypertarget{references}{%
\chapter*{References}\label{references}}
\addcontentsline{toc}{chapter}{References}

\markboth{References}{References}

\hypertarget{refs}{}
\begin{CSLReferences}{0}{0}
\end{CSLReferences}



\end{document}
